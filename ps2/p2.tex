\documentclass[11pt]{article}
%Gummi|065|=)
\title{\textbf{CS 374 Spring 2018\\Homework 2}}
\author{Nathaniel Murphy (njmurph3@illinois.edu)\\
		Tanvi Modi (tmodi3@illinois.edu)\\
		Marianne Huang (mhuang46@illinois.edu)}
\date{}

\usepackage{a4wide}
\usepackage{amsmath}
\usepackage{amsthm}
\usepackage{amssymb}
\usepackage{mathtools}
\usepackage{scrextend}

\begin{document}

\maketitle
\section*{Problem 2}
Let us first define a few helper functions. \\
\\
\begin{itemize}
	\item Define $b_k:2^k\rightarrow\mathcal{P}(\Sigma_k)$ such that for a given binary string $s$ of length $k$,\\$b_k(s)=\{a\in\Sigma_k\hspace{1mm}|\hspace{1mm}s_a=1\}$. Notice that $b_k(s)\subseteq\Sigma_k$.
	\item Define $h_k:\{0,1\}\rightarrow\Sigma_k$, where
	\[h_k(i)=\begin{cases}
		\epsilon & \text{if }i=0 \\
		k & \text{if }i=1
	\end{cases}\]
	\item Define $c_b:2^k\rightarrow\Sigma_k^*$ inductively.
	\[c_k(w)=\begin{cases}
		\epsilon & \text{if }w=\epsilon \\
		h_{|k|}(a)\cdot c_{|k-1|}(u) & \text{if }w=au
	\end{cases}\]
\end{itemize}
The language requires that we remember if a symbol is encountered before we read the $\natural$ symbol so that we can process all symbols correctly after the $\natural$ symbol. Let us prove this statement. \\ \\
Let us create a fooling set
\[F=\{c_k(s)\hspace{1mm}|\hspace{1mm}s\in 2^k\}\]
Choosing arbitrary $i,j\in F\setminus\{0^k\},i\neq j$. Without loss of generality, let us assume that $|i|\geq|j|$ (swap them if necessary). We see that the set $\big(b_k(i)\setminus b_k(j)\big)$ must be nonempty because no symbols in $i$ or $j$ repeat within each string and $|i|\geq|j|$. Let $w=\natural\cdot a,\hspace{1mm}a\in\big(b_k(i)\setminus b_k(j)\big)$. We see that $iw=i\natural a\in T_k$, but $jw=j\natural a\notin T_k$ because $a\in b_k(i)$, but $a\notin b_k(j)$. \\ \\
$|F|=2^k$, so we see that $M$, the DFA representing $T_k$ must have at least $2^k$ elements. \\ \\
\clearpage \ \\
Let us now prove a stronger statement. Denote:
\[L_b=\{c_k(s)\hspace{1mm}|\hspace{1mm}s\in 2^k\}\text{ and }L_{b\natural}=\{c_k(s)\cdot\natural\hspace{1mm}|\hspace{1mm}s\in 2^k\}\]
Let $F'=L_b\cup L_{b\natural}$. We have 4 cases for any $i,j$ chosen from $F'$: \\
\begin{addmargin}[2em]{0em}
	\underline{\textbf{Case $i,j\in L_b,\hspace{1mm}i\neq j$}:} Without loss of generality, let $|i|\geq|j|$ (swap them if necessary). Above, we have chosen a $w=\natural\cdot a\in\big(b_k(i)\setminus b_k(j)\big)$ such that $iw\in T_k$ and $jw\notin T_k$. \\ \\
	\underline{\textbf{Case $i\in L_b,j\in L_{b\natural}$}:} Let $j=j'\natural$. Let $w=\natural i$. It is clear that $iw=i\natural i\in T_k$, but $jw=j'\natural\natural i\notin T_k$. \\ \\
	\underline{\textbf{Case $i\in L_{b\natural},j\in L_b$}:} Same case as above, but reverse $i$ nad $j$. \\ \\
	\underline{\textbf{Case $i,j\in L_{b\natural}$}:} Notice that $i,j$ can be written in the form $i'\natural,j'\natural$. Without loss of generality, let $|i'|\geq|j'|$ (swap them if necessary). We see that the set $\big(b_k(i')\setminus b_k(j')\big)$ must be nonempty because no symbols in $i$ or $j$ repeat within each string and $|i'|\geq|j'|$. Choose $w=a\in\big(b_k(i')\setminus b_k(j')\big)$. It follows that $iw=i'\natural a\in T_k$ because $a\in b_k(i')$, while $jw=j'\natural a\notin T_k$ because $a\notin b_k(j')$.
\end{addmargin}
\ \\
We have been able to produce a $w\in\Sigma_k^*\cup\{\natural\}$ such that $\forall\hspace{1mm}i,j\in\Sigma_k^*\cup\{\natural\}$, either $iw\in T_k\wedge jw\notin T_k$ or $iw\notin T_k\wedge jw\in T_k$. \\ \\
Notice that $F'=L_b\cup L_{b\natural}$. and $L_b\cap L_{b\natural}=\emptyset$ trivially, $|F'|=|L_b|+|L_{b\natural}|=2^k+2^k=2\cdot 2^k=2^{k+1}$, so we now see that $M$, the DFA representing the language $T_k$ must have at least $2^{k+1}$ states. \\ \qed
\end{document}
