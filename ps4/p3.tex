\documentclass[11pt]{article}
\title{\textbf{CS 374 Spring 2018\\Homework 4}}
\author{Nathaniel Murphy (njmurph3@illinois.edu)\\
			Tanvi Modi (tmodi3@illinois.edu)\\
			Marianne Huang (mhuang46@illinois.edu}
\date{}
\usepackage{a4wide}
\usepackage{amsfonts}
\usepackage{amsmath}
\usepackage{amssymb}
\usepackage{amsthm}
\usepackage{stmaryrd}
\usepackage[margin=0.5in]{geometry}

\begin{document}
\maketitle
\section*{Problem 3}
Let us construct a Turing Machine $M_G$ to check whether a given string $w$ belongs to the language $ L(G)$, where $G=(\Sigma_G,\Gamma_G,R_G,S_G)$ is a context free grammar. \\ \\
Let us define a few functions that will help in defining $M_G$. ($\mathbb{W}$ represents the set of whole numbers)
\begin{itemize}
	\item Define $\#_{ops}:\hspace{1mm}R_G\rightarrow\mathbb{W}\text{ such that }\#_{ops}(\gamma)=\text{ \# distinct options in non-terminal }\gamma$
	\item Define a bijection $c:\hspace{1mm}(R_G,\mathbb{W})\rightarrow w$, where $w\in(\Sigma_G\cup\Gamma_G)^*$
	\item Define a bijection $r:\hspace{1mm}\mathbb{W}\rightarrow R_G$, such that $r(0)=S_G$
\end{itemize}
Let us now define $M_G=(\Gamma,\boxempty,\Sigma,Q,s,\text{accept},\text{reject},\delta)$:
\begin{itemize}
	\item $\Gamma=\Sigma_G\cup\Gamma_G\cup\{\rhd,\boxempty\}$
	\item $\boxempty=\boxempty$
	\item $\Sigma=\Sigma_G$
	\item $Q=\{\text{start, rt, accept, reject, temp}\}\cup\{(w,i)\hspace{1mm}|\hspace{1mm}0\leq i\leq|w|\}\cup\{(w_{ij},k)\hspace{1mm}|\hspace{1mm}w_{ij}=c(r(i),j),0\leq k<|w_{ij}\}$
	\item $s=$ start
	\item accept = accept
	\item reject = reject
	\item Let us now define $\delta$:
	\[\delta(q,t_1,t_2)\begin{cases}
		\big((w,0),(t_1,0),(\rhd,1)\big) & \text{if }q=\text{ start} \\
		\big((w,i+1),(w[k-i],-1),(t_2,0)\big) & \text{if }q=(w,i),\hspace{1mm}k=|w|,\hspace{1mm}i<|w| \\
		\big(\{(w_{0j},0)\hspace{1mm}|\hspace{1mm}0\leq j<\#_{ops}(S_G)\},(t_1,0),(\boxempty,0)\big) & \text{if }q=(w,i),\hspace{1mm}i=|w| \\
		\big(\{(w_{ij},0)\hspace{1mm}|\hspace{1mm}0\leq j<\#_{ops}(t_2)\},(t_1,0),(\boxempty, 0)\big) & \text{if }t_2\in\Gamma_G \\
		\big((w_{ij},k+1),(t_1,0),(w_{ij}[l-k],1)\big) & \text{if }q=(w_{ij},k),\hspace{1mm}\ell=|w_{ij}|,\hspace{1mm}k<\ell \\
		\big(\text{temp},(t_1,0),(\boxempty,-1)\big) & \text{if }q=(w_{ij},k),\hspace{1mm}k=|w_{ij}| \\
		\big(\text{rt},(t_1,1),(\boxempty,-1)\big) & \text{if }q\in\{\text{temp, rt}\},\hspace{1mm}t_2\in\Sigma\hspace{1mm}\wedge\hspace{1mm}t_1=t_2 \\
		\big(\text{reject},(t_1,0),(t_2,0)\big) & \text{if }t_2\in\Sigma\hspace{1mm}\wedge\hspace{1mm}t_1\neq t_2 \\
		\big(\text{accept},(t_1,0),(t_2,0)\big) & \text{if }t_1=\boxempty\hspace{1mm}\wedge\hspace{1mm}t_2=\rhd
	\end{cases}\]
\end{itemize}
Notice that transitioning into a $\{(w_{ij},0)\}$ state may spawn multiple threads if $\#_{ops}(R)>1$ for a non-terminal $R$.
\clearpage
When defining the states, the \{start, accept, reject\} states are trivial to understand, however, some explanation may be required for the \{temp, rt\} states. The temp state is the state that the machine goes into right after it has finished writing an entire string from a non-terminal to the tape. This makes it so that if the last letter it wrote was a terminal symbol, the machine will enter the \{rt\} state, and if the last symbol that it writes to the tape is a non-terminal, it will go back into a \{$(w_{ij},k)$\} state, the set of states used to non-deterministically write the non-terminals. The rt state is just for reading non-terminals from tape 2 and comparing them with the non-terminals on tape 1. Notice that it advances tape 1's head by 1 and tape 2's head by negative 1 because the string written to tape 2 is inverted. With that being said, notice the way that the machine writes a non-terminal to tape 2. It writes it backwards to ensure that the beginning of that non-terminal is at the end of the tape, because that is how the algorithm was designed. Finally, being in a $(w,i)$ state means that the machine has written the first i characters of the input string to tape 1 and will continue to do so. \\ \\
The functions defined at the top are only to simplify my definitions of the states. $\#_{ops}$ simply maps a rule to a whole number, that whole number being the number of paths (for lack of a better word) to choose from that rule. $c$ intuitively 'indexes' the options within a rule $r$ so that they may be accessed by using whole number indices. And finally, $r$ indexes the rules of a context free grammar, so that they may be accessed by using whole number indices.
\end{document}