\documentclass[11pt]{article}
%Gummi|065|=)
\title{\textbf{CS 374 Spring 2018\\Homework 3}}
\author{Nathaniel Murphy (njmurph3@illinois.edu)\\
		Tanvi Modi (tmodi3@illinois.edu)\\
		Marianne Huang (mhuang46@illinois.edu)}
\date{}

\usepackage{a4wide}
\usepackage{amsmath}
\usepackage{amsfonts}

\begin{document}

\maketitle

\section*{Problem 1}
\section*{1.}
$\Sigma=\{a,b\}$ and $\Delta=\{0,1\}$. Define homomorphism $h:\Sigma\rightarrow\Delta^*$ where $h(a)=01$ and $h(b)=10$. We see that $h^{-1}(01)=a$ and $h^{-1}(10)=b$.
\begin{itemize}
	\item $h^{-1}(\{0101\})=\{aa\}$
	\item $h^{-1}(\{00\})=\emptyset$
	\item $h^{-1}(\{001\})=\emptyset$
	\item $h^{-1}(\{1001\})=\{ba\}$
\end{itemize}
Notice that the second and third are equivalent to $\emptyset$ because
\[h^{-1}(L)=\{w\in\Sigma^*\hspace{1mm}|\hspace{1mm}h(w)\in L\}\Rightarrow h^{-1}(w)=\{u\in\Sigma^*\hspace{1mm}|\hspace{1mm}h(u)=w\}\]
and no such $u\in\Sigma^*$ exists such that $h(u)\in\{00,001\}$.
\\[15mm]
Let $L=L\big((00+1)^*\big)$.
\begin{itemize}
	\item $h^{-1}(L)=(ab)^*$
	\item $h(h^{-1}(L))=(1001)^*$
\end{itemize}
$h^{-1}((00+1)^*)=(ab)^*$ because for every $w\in h^{-1}((00+1)^*)$, $w$ must start with a 1, then must be followed by 00, then must be followed by another 1.
\clearpage
\section*{2.}
We will use the property that $h(uv)=h(u)h(v)$.
\subsection*{(a)}
For every $w\in\Sigma^*,\hspace{1mm}\delta^*_N(s',w)=\delta^*_M(s,h(w))$.
\subsection*{(b)}
\underline{Claim:} $\forall\hspace{1mm}w\in\Sigma^*,\hspace{1mm}\delta^*_N(s',w)=\delta^*_M(s,h(w))$. \\
\underline{Proof:} \\ \\
\underline{\textbf{Base}:} $|w|=0,\hspace{1mm}w=\epsilon$. $\delta^*_N(s',w)=\delta^*_N(s',\epsilon)=\delta(s',\epsilon)=s'=s=\delta'(s,\epsilon)=\delta'(s,h(\epsilon))=\delta^*_M(s,h(\epsilon))=\delta^*_M(s,h(w))$. \\ \\
\underline{\textbf{Inductive Hypothesis}:} Assume that $\forall\hspace{1mm}w\in\Sigma^*$, $|w|<k\Rightarrow\delta^*_N(s',w)=\delta^*_M(s,h(w))$. \\ \\
\underline{\textbf{Inductive Case}:} Let $w\in\Sigma^*$ such that $|w|=k$. $w$ can be written as $w=ua$, where $u\in\Sigma^{k-1}$ and $u\in\Sigma$. We see that:
\[\delta^*_N(s',w)=\delta^*_N(s',ua)=\delta^*_N\big(\delta^*_N(s',u),a\big)=\delta'\big(\delta^*_N(s',u),a\big)\]
\[=\delta'\big(\delta^*_M(s,h(u)),a\big)=\delta^*_M\big(\delta^*_M(s,h(u)),h(a)\big)=\delta^*_M(s,h(u)h(a))\]
Because $h(uv)=h(u)h(v)$, $u,v\in\Sigma^*$, it follows that
\[\delta^*_M(s,h(u)h(a))=\delta^*_M(s,h(w))\]
\subsection*{(c)}
Prove $L(N)=h^{-1}(L)$. \\ \\
Fix $w\in L(M)$. We want to show that $h^{-1}(w)\in L(N)$. \\
From the definition of $h^{-1}(L)=\{w\in\Sigma^*\hspace{1mm}|\hspace{1mm}h(w)\in L\}$, we see that it suffices to fix $w\in L(N)$ and show that $h(w)\in L(M)$. \\ \\
Because $Q'=Q$ and $A'=A$, we see that
\[\delta^*_N(s',w)=\delta^*_M(s,h(w)), \hspace{2mm}\forall\hspace{1mm}w\in\Sigma^*\]
which means that $\delta^*_N(s',w)\in A\Rightarrow\delta^*_M(s,h(w))\in A$. \\ \\
It follows that $L(N)=h^{-1}(L)$.
\end{document}
